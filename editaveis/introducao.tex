\chapter[Introdução]{Introdução}

\section{Contexto}

Durante as últimas décadas, houve diversas mudanças na estrutura populacional brasileira. Dentre essas mudanças, iniciou-se uma transição demográfica, que consiste na redução na mortalidade seguida por uma redução na taxa da natalidade \cite{travassos_elderly_2020}. Essa transição ocasionou no envelhecimento populacional, de modo que houve uma diminuição da população em idade ativa e um envelhecimento da estrutura etária, resultando no aumento da população de pessoas idosas no país \cite{vasconcelos_transicao_2012}. Segundo \citeonline{kurniawan_research-derived_2005}, as pessoas idosas podem apresentar dificuldades na realização de tarefas que exigem habilidades visuais, auditivas e motoras. Além disso, a idade também pode afetar a capacidade de aprendizado e a memória \cite{salthouse_when_2009}.

O uso de Tecnologias da Informação e Comunicação (TICs) tem sido cada vez mais frequente. Após as restrições definidas pelas autoridades sanitárias para redução do contágio da COVID-19 e contenção da pandemia \cite{anjos_recomendacao_nodate}, a necessidade do uso das TICs ultrapassaram o ceticismo e a hesitação na sua utilização \cite{sixsmith_older_2022}. Todavia, a população idosa tende a ser mais heterogênea por conta da variedade de efeitos relacionados à idade \cite{rodrigues2018support}, de modo que é necessário pensar em políticas de Inclusão Digital para essa população \cite{bez_inclusao_2006}. 

A lei nº 10.741 de 1º de outubro de 2003 institui o Estatuto do Idoso, "destinado a regular os direitos assegurados às pessoas com idade igual ou superior a 60 (sessenta) anos" \cite{lei10741}. Segundo o artigo 3º do Estatuto, é prevista a "viabilização de formas alternativas de participação, ocupação e convívio da pessoa idosa com as demais gerações"   \cite{lei10741}. Em vista disso, é necessário adaptar as TICs para que elas possam ser utilizadas por pessoas idosas, garantindo que as diversas características de pessoas sejam capazes de utilizá-las.

Segundo \citeonline{bevan_quality_1999}, qualidade em uso é definida como a "visão do usuário sobre a qualidade de um sistema que contém software, e é medida em termos do resultado do uso do software". A \citeonline{ISO/IEC25010} define um modelo de qualidade do produto, composto de 8 características: Adequação funcional, Desempenho e Eficiência, Compatibilidade, Usabilidade, Confiabilidade, Segurança, Manutenibilidade e Portabilidade. Além disso, a qualidade do produto influencia diretamente na qualidade em uso \citeonline{bevan_quality_1999}. Portanto, a qualidade do produto de software é essencial para garantir a qualidade em uso. 

\section{Justificativa}

A transição demográfica iniciou-se na década de 40 e é marcada por uma alteração na estrutura populacional, sendo caracterizada por baixas taxas de natalidade e mortalidade \cite{omran_epidemiologic_2005}. Deste modo, houve no Brasil um aumento da população idosa, essa representando 14,7\% da população em 2021 \cite{ibge_caracteristicas_gerais}. Historicamente, a população idosa apresenta dificulade na utilização de tecnologias da informação e comunicação, sofrendo de um processo de exclusão digital \cite{diniz_inclusao_2020}. Segundo dados da PNAD Contínua - Tecnologia da Informação e Comunicação de 2021 (TIC), 57,5\% das pessoas com 60 anos ou mais utiliza a internet, em comparação a 44,8\% da mesma pesquisa realizada no ano de 2019 \cite{ibge_acesso_a_Internet}. Durante a pandemia de COVID-19, houve um aumento na utilização de TICs por pessoas idosas, com as redes sociais sendo o maior exemplo de uso \cite{sixsmith_older_2022}.

Portanto, diante do aumento do uso de tecnologia por essa parcela da população em decorrência da pandemia e do envelhecimento populacional, tornou-se necessário analisar a qualidade das tecnologias levando em consideração a perspectiva do público idoso, a fim de compreender quais características de qualidade do produto ocasionam na dificuldade de uso das tecnologias. Ao considerar a perspectiva do público idoso na concepção de tecnologias e ao promover a inclusão dessa parcela da população, podemos criar soluções mais eficazes e acessíveis. Essa abordagem não apenas reflete uma responsabilidade social, mas também contribui para a evolução da ciência e da engenharia, impulsionando a sociedade em direção a um futuro mais inclusivo e equitativo.

\section{Questão de Pesquisa}
Com base no contexto e justificativa especificados no trabalho, foi definida a seguinte questão de pesquisa: \textbf{Quais características de qualidade, conforme definidas pela ISO/IEC 25010, estão associadas a cada perfil de idosos, ao considerar as características de qualidade do produto e os dados da TIC Domicílios?}

\section{Objetivos}
Este trabalho foi estruturado com um objetivo geral, que foi subdividido em objetivos específicos para facilitar a condução adequada do projeto.

\subsection{Objetivo geral}
O objetivo geral deste trabalho é \textbf{analisar o perfil das pessoas idosas, considerando as características de qualidade do produto, com base nos dados da TIC Domicílios}. Deste modo, espera-se entender quais características de qualidade, segundo a ISO/IEC 25010, estão associadas a cada perfil de idosos.

citeonline\subsection{Objetivos específicos}
\textbf{OE1} - Compreender o que é Modelo de Qualidade de Software;

\textbf{OE2} - Perceber as dimensões associadas ao modelo de Qualidade de Software;

\textbf{OE3} - Entender os dados da base de dados TIC Domicílios;

\textbf{OE4} - Vincular as variáveis que estão disponíveis na base de dados com as características de Qualidade de Software;

\textbf{OE5} - Aplicar Análise Fatorial nos perfis de idosos;

\textbf{OE6} - Comparar os resultados das análises entre os perfis de idosos;

\textbf{OE7} - Identificar os principais fatores que influenciam os perfis de idosos com base nos dados coletados.

\section{Organização do Trabalho}

Este trabalho está organizado da seguinte forma:

\textbf{Capítulo 1: Introdução -} Neste capítulo, é apresentado o contexto da pesquisa, seguido da justificativa, da questão de pesquisa e dos objetivos gerais e específicos.

\textbf{Capítulo 2: Referencial Teórico -} Este capítulo aborda os trabalhos utilizados como referência em relação à pessoa idosa e ao uso de tecnologias por essa população, bem como a transição demográfica e a qualidade de software.

\textbf{Capítulo 3: Metodologia -} Aqui, descreve-se a classificação da pesquisa, apresentam-se as ferramentas utilizadas e o cronograma planejado, além da fonte de dados estudada e do método de análise fatorial utilizado.

\textbf{Capítulo 4:  Resultados Esperados -} Neste capítulo, são apresentados os resultados esperados ao final da pesquisa.