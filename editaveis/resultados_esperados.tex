\chapter[Resultados esperados]{Resultados esperados}

O trabalho apresentado tem como expectativa oferecer uma experiência de aprendizado aprofundada sobre o modelo de qualidade de software, incluindo suas características e subcaracterísticas definidas pela \citeonline{ISO/IEC25010}, assim como identificar as características de qualidade do produto de software que estão associadas a cada perfil de idosos, para entender os principais obstáculos que dificultam a utilização de tecnologias por parte dessa população.

Para atingir esse objetivo, será aplicado o método de análise fatorial nos dados coletados da TIC Domicílios para identificar os principais fatores que influenciam os perfis de idosos, agrupando as características de qualidade do software em fatores comuns e facilitando a compreensão das relações entre elas. Será realizada uma comparação dos resultados obtidos nas análises fatoriais entre os diferentes perfis de idosos, permitindo identificar padrões e diferenças nas características de qualidade do software que afetam cada grupo de idosos. Isso fornecerá informações valiosas para o desenvolvimento de tecnologias mais acessíveis e amigáveis aos idosos.

Em resumo, este trabalho pretende fornecer \textit{insights} sobre as características de qualidade do software que estão associadas à dificuldade de uso das tecnologias por parte da população idosa. Os resultados esperados contribuirão para o desenvolvimento de tecnologias mais inclusivas e adaptadas às necessidades desse grupo específico, visando promover a inclusão digital e melhorar a experiência do usuário idoso.