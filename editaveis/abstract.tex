\begin{resumo}[Abstract]
 \begin{otherlanguage*}{english}
The aging population has been noticeable in Brazil for several years. The elderly population faces difficulties in using Information and Communication Technologies (ICTs). As this population grows, new challenges arise in the development of ICTs that adequately meet their needs. During the COVID-19 pandemic, the need to use Information and Communication Technologies (ICTs) has intensified due to the restriction of social interactions imposed by the implemented social isolation measures. This has made ICTs an essential tool for social interaction. However, various available technologies have quality issues that hinder their use by older people, creating a "digital divide" environment. In light of this, this study proposes a descriptive research aiming to analyze the profile of older people, considering the characteristics and sub-characteristics of product quality based on data from the TIC Domicílios survey conducted by Cetic.br. The analysis will be carried out using the statistical technique of Factor Analysis, by separating the profiles of interest and examining which characteristics and sub-characteristics of quality have the greatest impact on the difficulty of use among the various profiles of elderly individuals.

   \vspace{\onelineskip}
 
   \noindent 
   \textbf{Key-words}: elderly person, elderly, software quality, ICT
 \end{otherlanguage*}
\end{resumo}
