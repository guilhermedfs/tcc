\begin{resumo}
O envelhecimento populacional já é notado no Brasil há alguns anos. A população idosa apresenta dificuldade na utilização de Tecnologias da Informação e Comunicação (TICs). Com isso, conforme essa população cresce, surgem novos desafios quanto ao desenvolvimento de TICs que a atendam adequadamente. Durante a pandemia de COVID-19, a necessidade de utilizar TICs se intensificou, uma vez que as interações sociais foram restritas devido ao isolamento social implementado para conter o vírus. Isso tornou as TICs uma ferramenta indispensável para o convívio social. Diversas tecnologias disponíveis apresentam problemas de qualidade que dificultam sua utilização por pessoas idosas, criando um ambiente de "exclusão digital". Diante disso, este trabalho propõe uma pesquisa descritiva, onde o objetivo principal é analisar o perfil das pessoas idosas, considerando as características e subcaracterísticas de qualidade do produto, com base nos dados da pesquisa TIC Domicílios realizada pela Cetic.br. A análise será realizada por meio da técnica estatística de Análise Fatorial, a partir da separação dos perfis de interesse e análise de quais características e subcaracterísticas de qualidade mais impactam na dificuldade de uso pelos diversos perfis de idosos.

 % O resumo deve ressaltar o objetivo, o método, os resultados e as conclusões 
 % do documento. A ordem e a extensão
 % destes itens dependem do tipo de resumo (informativo ou indicativo) e do
 % tratamento que cada item recebe no documento original. O resumo deve ser
 % precedido da referência do documento, com exceção do resumo inserido no
 % próprio documento. (\ldots) As palavras-chave devem figurar logo abaixo do
 % resumo, antecedidas da expressão Palavras-chave:, separadas entre si por
 % ponto e finalizadas também por ponto. O texto pode conter no mínimo 150 e 
 % no máximo 500 palavras, é aconselhável que sejam utilizadas 200 palavras. 
 % E não se separa o texto do resumo em parágrafos.

 \vspace{\onelineskip}
    
 \noindent
 \textbf{Palavras-chave}: pessoa idosa, idoso, qualidade de software, TIC
\end{resumo}
